\addcontentsline{toc}{chapter}{Abstract}
\begin{center}
\textbf{\large Abstract}
\end{center}

Nuclear-polarized $^{3}$He targets have been widely used in electron-scattering experiments in Thomas Jefferson National Accelerator Facility (JLAB) since mid 1990’s.  It is of great importance to produce large amounts of $^{3}$He gas with high polarization. 

The latest experiments run in JLAB prior to the 12GeV upgrade have been using cells polarized with Spin-Exchange Optical Pumping (SEOP). These cells were made of the GE180 glass and use a two-chambered design. The top chamber, known as the “pumping chamber”, is where $^{3}$He is polarized through SEOP. The bottom chamber, known as the “target chamber”, is where electron scattering occurs. Great effort has been made in our lab to develop this generation of cells. Alkali-hybrid SEOP together with narrowband laser diode arrays have increased the $^{3}$He polarization from 37\% to 65\%. Among other things, we also carefully studied an additional spin relaxation mechanism that limits the maximum achievable $^{3}$He polarization. 

The 12GeV upgrade makes the future experiments much more demanding in terms of target cell performance. One challenge it brings is the high relaxation due to electron beam. We have designed and tested a new style cell that uses convection instead of diffusion to increase the rate at which the polarization in the target chamber is being replenished by gas from pumping chamber. We have obtained over 50\% polarization with controllable convection speed so far.

An additional problem that comes with higher beam current is that the glass end windows of traditional design are not likely to survive the experiments. Our group started exploring the option of using metal end windows from a decade ago. The first problem to solve is to find out the correct material and the proper technique to incorporate metal without introducing significant spin relaxation and still being able to hold high pressure gas (12 atm) inside. This is a brand new technique that may have a profound impact of future cell designs once fully developed. Although no metal end windows have been tested so far, multiple glass cells with different kinds metal tubes (much larger in area compared to the end windows that will be used in JLAB experiments) attached were examined and were enough to convince us the extra spin relaxation is not likely to cause significant problems. The metals tubes were connected to Pyrex glass with knife-edge (housekeeper) seals and stayed intact through high pressure tests. After exploring options such as pure copper, gold coated copper, titanium, stainless steel, gold coated titanium, we have established that electroplating gold on copper substrate yields the best result so far. Further tests are planned before attaching metal end windows to GE180 glass and using them in electron-scattering experiments.

