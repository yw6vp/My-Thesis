\addcontentsline{toc}{chapter}{Abstract}
\begin{center}
	\textbf{\large Abstract}
\end{center}

Historically, $^3$He targets for electron scattering experiments have been polarized through spin-exchange optical pumping (SEOP). Polarized laser light passes its circular polarization to alkali metal vapor, which then transfers its polarization to $^3$He through spin-exchange collisions. 

This thesis discusses the basics of SEOP and the polarimetry techniques used in our lab. Narrowband laser and alkali-hybrid SEOP have improved the performance of targets significantly. In alkali-hybrid SEOP, potassium is used together with rubidium for transferring polarization to $^3$He nuclei. We discussed the data collected over many pure-rubidium targets and alkali-hybrid targets. In the course of analyzing the data, we also studied the ``X factor" which limits the highest achievable polarization of $^3$He.

Because the experiments planned for the 12GeV era in Jefferson National Laboratory (JLAB) will use much higher electron beam current, we are exploring the possibility of using metal (instead of glass) as the entry points (commonly referred to as ``end windows") for future targets. We established the metal composition and developed the techniques to incorporate metal to targets without introducing significant spin-relaxation rates. We have successfully demonstrated that future targets can be constructed with metal end windows and are very close to making such targets.