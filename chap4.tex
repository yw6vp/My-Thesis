\chapter{Hybrid Studies}
\label{chap4}

\section{Overview}

In the first part of this chapter, I present the development of high-performance polarized $^{3}$He targets for use in electron scattering experiments that utilize the technique of alkali-hybrid spin-exchange optical pumping. Data of 24 separate target cells are presented, each of these cells was constructed while preparing of one of four experiments at Jefferson Laboratory. The results document dramatic improvement in the performance of polarized $^{3}$He targets. I focus on the data analysis work in this chapter since most of the data had already been taken by the time I joined the group. Other details are described by Jaideep Singh~\cite{PhysRevC.91.055205}. With the wide range of data, we successfully determined the so-called X-factors that quantify a temperature-dependent and as-yet poorly understood spin-relaxation mechanism that limits the maximum achievable $^{3}$He polarization to well under 100\%. We developed a simulation of the alkali-hybrid spin-exchange optical pumping process that provided a means to calculate volume averaged alkali polarization. The data collected also served as a measurement of the K-$^{3}$He spin-exchange rate coefficient $k_{se}^{K}=(7.46\pm0.62)\times10^{-20}$ cm^{3}/s over the temperature range 503 K to 563 K.

In the second part of the chapter, I report the results we have so far on developing the next generation target cell. As mentioned in previous chapters, target cells are composed of two distinct chambers: a pumping chamber (PC) where gas is polarized, and a target chamber through which the electron beam passes. The two chambers are traditionally connected by a single transfer tube. The polarization in the target chamber is replenished by gas from pumping chamber through diffusion. This is not a problem as long as the time scales associated with diffusion are short compared with the time scales associated with the depolarization process. The time scale of diffusion through transfer tube is roughly 0.5-1 hour. Due to the increasing electron beam current, the polarization gradient between the two chambers increased from 1-2\% to up about 8\% in more recent experiments. Future experiments are likely to use beam currents 4 times or more of what caused 8\% polarization gradient. A new design was developed Peter Dolph which circulates the gas with convection instead of diffusion. Dolph demonstrated the design with a prototype cell, later I tested the first and second convection style cells with high $^{3}$He polarization while having convection on. The success with our new design enables future experiment to run with higher electron beam current without causing too much polarization gradient.

\section{}