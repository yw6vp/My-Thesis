\chapter{Development of Hybrid Targets}
\label{chap5}

\section{Overview}

Electron scattering experiments at JLab have traditionally used $^{3}$He targets made of glass due to the compatibility with spin-exchange optical pumping, the capability to be shape into desired geometries through glass blowing, and the excellent nuclear spin relaxation properties. The borosilicate glass Pyrex had been the glass of choice prior to the discovery that the much less permeable aluminosilicate glass generally provides longer spin-relaxation time. The most recent targets were thus composed entirely of the aluminosilicate glass GE180.

After the 12 GeV upgrade, JLab plans to run experiments with much higher electron beam currents. The maximum current used before the upgrade was 15$\mu$A, while future experiments will be run at up to 60$\mu$A. We believe an all-glass target cell might survive long enough for an experiment with 30$\mu$A, but it is unlikely to survive at 60$\mu$A. A natural solution would be to replace the thin glass window (where electron beam enters and exits the target cell) with a material with higher strength. 

Deninger~\emph{et al.} from the Mainz group showed relaxation time of various metal surfaces: Mg (6 h), Al (6 h), Zn(12 h) etc. Gold caught our attention in particular, because it has a relatively long relaxation time of 20 h. This relaxation time was measured by coating the glass surface with gold, thus the area of gold surface was much larger than what will be needed for target windows. In addition, while the coating process made sure of the chemical purity, it did not make effort in ensuring the microscopic smoothness, which means the surface area was further increased. In light of this, our group have tested 19 cells with various geometries and materials, most of which incorporate a OFHC (Oxygen-free high thermal conductivity) copper tube with gold coating. OFHC copper was chosen as the substrate in most cases because of its structural strengths and the familiarity with manufacturers. Towards the end of my work, we achieved a 15.6 h relaxation time with a Pyrex cell that had a 5'' long by 1'' gold coated copper tube attached horizontally. By extrapolating the relaxation rate due to gold surface from this result, we believe the relaxation rate introduced by small metal windows in a target cell will be less than 1/135 h$^{-1}$.

\section{Wall Relaxation of $^{3}$He}



